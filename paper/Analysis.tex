\section{Results}
\label{Results}
%\section{Data Analysis}
%\label{Data Analysis}

\subsection{Flight Data}
%Notes on how to take the results...
%What is being presented: Five plots total, the flight profile for each year, and cumulative dose for each flight.

\begin{figure}[H]
\centering
\includegraphics[scale=.45]{count_rate_altitude_2017.pdf}
\caption{Count Rate and Altitude vs Time for 2017 flight}
\label{fig:ratealttime_2017}
\end{figure}
%
%First:
%Present and characterize the first two plots - the data of the flight with counts vs altitude vs time.  Here we can talk about each flight and data shown.  Make sure to mention how the count rate changed as time and altitude changed.  We can talk about the flight time, the flight altitude variance, the coasting altitude, and how it all compares to the count rate.  We can be as descriptive as we can.
%mention the position of the pfotzer-regener maximum for both years.  Talk about discrepancies as the data shows.  Speculate in discussion later.
Both HASP balloons containing the SORA payloads were launched separately on September 4th on 2017 and 2018 respectively.  The MiniPIX was set to operate in Time over Threshold mode with a bias voltage of 4 Kev.  Frames were collected at static 4 second intervals during the 2017 flight and between 1 and 4 second intervals for the 2018 flight.  The complete flight altitude profiles with total count rates are shown in Figure~\ref{fig:ratealttime_2017} for 2017 and Figure~\ref{fig:ratealttime_2018} for 2018.
%
\begin{figure}[H]
\centering
\includegraphics[scale=.50]{count_rate_altitude_2018.pdf}
\caption{Count Rate and Altitude vs Time for 2018 flight}
\label{fig:ratealttime_2018}
\end{figure}
%
As shown in Figure~\ref{fig:ratealttime_2017} and Figure~\ref{fig:ratealttime_2018}, both flights have very similar flight profiles.  It took each flight about 2 hours and 30 minutes to ascend to a stable float altitude.  It is important to mention that the 2017 flight reach and maintained a float altitude of approximately $\SI{31.5}{\kilo\meter}$.  In contrast, the 2018 flight kept a float altitude of $\SI{37.2}{\kilo\meter}$.  The rate of decent was slow and steady for both flights.  As such, data was collected continuously throughout the whole flights.
%
\begin{figure}[H]
\centering
\includegraphics[width=\textwidth]{temps_flight_2018.pdf}
\caption{2018 recorded temperatures of MiniPIX and Raspberry Pi Flight computer.}
\label{fig:temps_2018}
\end{figure}
%
The MiniPIX and RPI flight computer (Raspberry Pi) both operated well throughout both flights.  The 2018 flight was better equipped to record system health and operation status.  A key status indicator of system health was the system temperature.  These temperatures recorded are shown in Figure~\ref{fig:temps_2018}.  The MiniPIX temperature stayed within expected ranges and never peaked beyond $\SI{40}{\degreeCelsius}$.  

Likewise, the RPI stayed with operational temperatures.
With both flights reaching altitudes beyond $\SI{25.0}{\kilo\meter}$, the Pfotzer-Regener maximum was clearly observed.  In 2017, the Pfotzer-Regener maximum peaked at approximately  $\SI{18}{\kilo\meter}$.  Similarly, the 2018 mission recorded Pfotzer-Regener maximum peaking around $\SI{20}{\kilo\meter}$. After these peaks in count rate, the counts per minute tapered off and varied inversely with altitude.  These count rates then remained constant throughout the remainder of the flights.
%FOR Discussion - compare to BEXUS flights and RADX flight as well, and mention how the maximum is similar/different.  Here the maximum in the Hands et al RaD-X mission found the Pfotzer maximum around 60,000 ft (~18 km).  
%
%
%
\begin{figure}[H]
\centering
\begin{subfigure}{.5\textwidth}
  \centering
  \includegraphics[scale=.45]{dva_stderr.pdf}
  \caption{Dose rate in silicon vs. altitude.}
  \label{fig:sub1}
\end{subfigure}%
\begin{subfigure}{.5\textwidth}
  \centering
  \includegraphics[scale=.45]{cva_stderr.pdf}
  \caption{Detector Hits vs. altitude.}
  \label{fig:sub2}
\end{subfigure}
\caption{Figure ~\ref{fig:sub1} shows the absorbed dose rate per hour as a function of altitude from the MiniPIX.  Figure ~\ref{fig:sub2} shows the counts per minute as a function of altitude again from the MiniPIX data.}
\label{fig:sub2}
\end{figure}
%Next, go into the absorbed rate vs altitude plot.  Mention how the LET varies for materials such as silicon and muscle.  Go into the pfotzer-regener maximum again, compare.  Compare this to the accompanying figure, Detector Hits vs Altitude.  There is a slight discrepancy in the flights, this may be useful to mention.  All error bars are 1 sigma standard deviation.
The Pfotzer maximum can again be observed in the absorbed dose rate in silicon - Figure~\ref{fig:sub1}.  This peak again is clearly seen at an altitude of $\SI{18}{\kilo\meter}$ for 2017.  In comparison to Figure~\ref{fig:ratealttime_2018} for 2018, the peak can be observed higher in altitude as shown in Figure~\ref{fig:sub1}.  There is however, a large amount of error that must be taken into account for measurements beyond $\SI{15}{\kilo\meter}$ of altitude. 
%Another area for the discussion - great area to talk about the discrepancies.  Here we only talk about data and thats it.
Similarly, Figure~\ref{fig:sub2} again confirms the Pfotzer maximum reached for both years.
%For discussion: Here the error is more consistent and Pfotzer maximum is reached at the expected altitudes.
\begin{figure}[H]
\centering
\includegraphics[width=\textwidth]{ctva-cropped.pdf}
\caption{Distribution of Cluster Type Counts vs. altitude for the duration of the 2018 flight.}
\label{fig:cluster2018}
\end{figure}
%Finally, the last figure Cluster Type Counts vs Altitude.  This is useful due to the MiniPIX being able to analyze indivudual track lenghts.  From here, these tracks can be characterized into different and indidivual categories.  This is useful for LET calculations and overall more precise for dosage calculations.  It may also help with particle identification.
The distribution of cluster types as they vary with altitude is shown in Figure~\ref{fig:cluster2018}.  
\begin{figure}[H]
\centering
\includegraphics[width=\textwidth]{tracks.png}
\caption{Frame collected at float.}
\label{fig:frame1}
\end{figure}
Figure~\ref{fig:frame1} shows one of many particle tracks recorded during the duration of both 2017 and 2018 flights.  The track lengths of ionizing particles can be calculated and used to directly measure the linear energy transfer of various primary and secondary particles.


