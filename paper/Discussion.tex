%Report results first, including figures, particle count, dose, altitude, and temperatures.
\section{Discussion}
\label{Discussion}
%Introduce the basic system again and how it was briefly setup.
%Discuss dose and particle count measurements as they relate to classical theory
%Talk about the success of the overall PAYLOAD briefly.
% Talk about cumulative dose somewhere.
A primary goal of the SORA flights was to successfully utilize the MiniPIX device in dosimetric applications.
The figures presented in Section \ref{Results} clearly demonstrate this success as the results from the two missions are validated by comparison to previous stratospheric balloon flights.
The BEXUS \cite{bexus} balloon flights have shown that Medipix sensors are able to perform in the stratosphere.
NASA's RaD-X missions \cite{rad-x} provided unprecedented results to which SORA's data can be compared.

\subsection{System Performance}
% Here its just specific discussion about the system performance.
% Talk about success of overall SYSTEM.  Include TEMP data about the flight and hardware of the system.  Mention altitude, humidity, and any atmospheric conditions available for both flights.
Due to the lack of heat convection in the stratosphere, a pressing concern regarding the functionality of the MiniPIX was the temperature of the device at float altitude. 
The device is guaranteed to operate within the manufacturer's limits of \SIrange{0}{70}{\celsius}.
With the use of a relatively compact heatsink, the temperature of the device remaned within \SIrange{10}{40}{\celsius}.
The success of the heatsink suggests that less material could be utilized, consequently reducing the overall size of the apparatus.

\subsection{Space Weather}
% Relevancy of space weather to commercial airlines.
% Astronauts' time in space is decreasing.
% Mention the issues with modeling aviation radiation exposure with simulations as opposed to real data. I found a paper from one of the RaD-X authors that details this point: https://agupubs.onlinelibrary.wiley.com/doi/10.1002/2016SW001399

\subsection{Success of SORA Relative to Others}
% Detail the results which match the NASA article
The success of the application of the MiniPIX as a dosimeter is clearly seen with direct comparison to the RaD-X missions. Figure \ref{fig:sub1}
The absorbed dose measured by SORA is slightly higher that that measured by the RAD-X.
This can be explained by the difference of the year of each flight and the relationship between the solar cycle and GCR flux.
The solar cycle is inversely related to GCR flux \cite{hathaway}, i.e. when the sun is at solar minimum the GCR flux is at a maximum.
The RaD-X mission was in 2015 during the decline of the current solar cycle, solar cycle \num{24}. 
Solar cycle 24 hit its solar minimum in March 2018, only a few months prior to the second SORA flight in September 2018.


% Explain how we expanded upon their work, but that there is still much more than can be done.
% Mention the need for a low-cost dosimeter and how our configuration is reaching that goal. The only setback in the high-cost of the commercially unavailable MiniPIX.
% Mars environment

% Should we have a ``Future Work'' section?
%\subsection{Future Work}
