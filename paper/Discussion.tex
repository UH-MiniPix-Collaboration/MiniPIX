%Report results first, including figures, particle count, dose, altitude, and temperatures.

\section{Discussion}

\label{Discussion}

%Introduce the basic system again and how it was briefly setup.
%Discuss dose and particle count measurements as they relate to classical theory
%Talk about the success of the overall PAYLOAD briefly.
% Talk about cumulative dose somewhere.
A primary goal of the SORA flights was to successfully utilize the MiniPIX device in dosimetric applications. The figures presented in Section \ref{Results} clearly demonstrate this success as the results from the two missions are validated by comparison to previous stratospheric balloon flights. The BEXUS \cite{bexus} balloon flights have shown that Medipix sensors are able to perform in the stratosphere. NASA's RaD-X missions \cite{rad-x} provided results to which SORA's data can be compared.

\subsection{System Performance}

% Here its just specific discussion about the system performance.

% Talk about success of overall SYSTEM.  Include TEMP data about the flight and hardware of the system.  Mention altitude, humidity, and any atmospheric conditions available for both flights.

Due to the lack of heat convection in the stratosphere, a pressing concern regarding the functionality of the MiniPIX was the temperature of the device at float altitude. The device is guaranteed to operate within the manufacturer's limits of \SIrange{0}{70}{\celsius}. With the use of a relatively compact heatsink, the temperature of the device remaned within \SIrange{10}{40}{\celsius}. The success of the heatsink suggests that less material could be utilized, consequently reducing the overall size of the apparatus.

\subsection{Space Weather}

% Relevancy of space weather to commercial airlines.

% Astronauts' time in space is decreasing.

% Mention the issues with modeling aviation radiation exposure with simulations as opposed to real data. I found a paper from one of the RaD-X authors that details this point: https://agupubs.onlinelibrary.wiley.com/doi/10.1002/2016SW001399


The radiation environment in the Earth-Moon system has become a cause for concern over the last several years as comparisons between the levels of solar radiation during the solar cycle have been made. The increase in length of the solar minimum phase, and subsequent decrease in duration of the solar maximum, corresponds to record high cosmic ray activity.  A comparison between the number of days a 30 year old male could spend in interplanetary space in the 1990's versus 2014 revealed a substantial decrease from 1000 days in the 1990's to approximately 700 in 2014. \cite{cosmic_flux}.

The collected data associated with radiation exposure to commercial flight attendants and pilots provides another reason to monitor cosmic ray activity.  The United States National Council on Radiation Protection Measurements reported, between 2003-2006, flight crews as having the highest average effective dose among terrestrial radiation workers.  And a 2015 epidemiology study revealed a 70 \% increase in the risk of miscarriages in flight attendants who received 0.1 mGy or more in their first trimester.  While there is a wealth of compelling data that strongly suggests a need to monitor radiation exposure, flight crews remain the only occupational group exposed to unqualified and undocumented levels over the duration of their careers. \cite{human_impact}.

While there are physics-based aviation radiation models, there are several major issues with even the most current models. Depending on the dosimetric quantities used, the variation in agreement between the measured and model data ranges from within 10\% to over 50\%. 
%this next sentence is super long - we need to slim it down or throw it into chunks.
Accurately representing primary cosmic ray particles incident at the top of the atmosphere, and subsequent secondary particle production interactions and transport processes at all altitudes above commercial flight levels, and the lack of measurements covering a range of atmospheric altitudes, latitudes, and solar cycle activity, are all necessary to render exposure predictions similar to those measured at aviation altitudes \cite{modeling}. 

Space radiation and GCRs pose real health risks to astronauts and pilots in any field.  With the increase in cosmic ray flux in the Earth Moon system, radiation exposure in LEO and the atmosphere must be monitored to ensure they do not exceed the occupational dose limits.  The ability to measure real-time exposure is arguably a crucial tool that can help mitigate increased health risks.  Additionally, Studying radiation types and dose levels near the nebulous border between space and Earth's atmosphere, and within the various zones of the atmosphere, facilitates preparation for current and future space missions; and aviation in general. 

%To monitor tnauts he radiation in the upper atmosphere and beyond, a USB TimePIX device could be effectively used as a low cost dosimeter.  TimePIX detectors have many applications in the realm of particle physics, specially with their small physical and power consumption footprint. Overall, TimePIX devices could be setup to create a network of devices to better understand the Earth space radiation environment and observe space weather.
% We've got this taken care of in the section below. 



\subsection{Success of SORA Relative to Others}

% Detail the results which match the NASA article
The success of the application of the MiniPIX as a dosimeter is clearly seen with direct comparison to the RaD-X missions. The RaD-X flights were characteristically very similar to the SORA flights in that both missions were launched from Fort Sumner, New Mexico, had similar float altitudes, and had similar flight durations. RaD-X utilized a RaySure monitor to measure the dose during the balloon flight. The MiniPIX is able to acquire data from the same variety of particles species as RaySure with the exception of neutron interactions. However, the MiniPIX is able to be modified with a scintillating material in order to indirectly detect neutrons, as shown in \cite{medipix-neutron-scintillator-1} and \cite{medipix-neutron-scintillator-2}. Unlike the RaySure, the MiniPIX is able to record the track length of incident particles. The LET spectrum can be calculated using the deposited particle energy measured by the detector as well as the track length of the energy deposition. Quality factors regarding the dose equivalent can extracted from the LET spectrum. Thus, the dose equivalent can be calculated only using data measured by the MiniPIX whereas conventional methods require the use of Monte Carlo simulations \cite{stuartthesis} or additional equipment.   

Figure \ref{fig:sub1} aligns beautifully with Figure 3 presented by RaD-X \cite{rad-x}. Both figures show the absorbed dose as a function of altitude and the Regener-Pfotzer Maximum is visible in all data sets. The absorbed dose measured by SORA is slightly higher that that measured by RaD-X. This can be explained by the difference of the year of each flight and the relationship between the solar cycle and GCR flux. The solar cycle is inversely related to GCR flux \cite{hathaway}, i.e. if the sun is at solar minimum then the GCR flux is at a maximum. The RaD-X mission was in 2015 during the decline of the current solar cycle, solar cycle \num{24}. Solar cycle \num{24} hit its solar minimum in March 2018, only a few months prior to the second SORA flight in September 2018. 

The results of the SORA missions touched upon various points highlighted at the conclusion of RaD-X. The scientists behind the RaD-X missions stressed the importance of a cheap and compact dosimeter, as the importance of dosimetric measurements are being realized beyond scientific applications. The quasi-commercial availability of the MiniPIX makes the device a strong candidate for such applications.

There is much to be learned beyond the applications of SORA. The radiation shielding provided by Earth's atmosphere at SORA's float altitude is similar to that provided the Martian atmosphere at the surface of Mars \cite{rad-x}. By better understanding the complex nature of the radiation field in the stratosphere, humanity can better prepare for our exploration on Mars. Further applications also includes dose measurements for aviation as the need for such use was outlined by our discussion of space weather.

% Explain how we expanded upon their work, but that there is still much more than can be done.

% Mention the need for a low-cost dosimeter and how our configuration is reaching that goal. The only setback in the high-cost of the commercially unavailable MiniPIX.

% Mars environment
