%Report results first, including figures, particle count, dose, altitude, and temperatures.
\section{Discussion}
\label{Discussion}
%Introduce the basic system again and how it was briefly setup.
%Discuss dose and particle count measurements as they relate to classical theory
% Maybe we can put the ``System performance'' section up here as an introduction to the discussion section?
%Talk about the success of the overall PAYLOAD briefly.
\subsection{System Performance}
% Is this section needed in the discussion or can we just slip this information in when we mention the plots?
% Here its just specific discussion about the system performance.
% Talk about success of overall SYSTEM.  Include TEMP data about the flight and hardware of the system.  Mention altitude, humidity, and any atmospheric conditions available for both flights.
\subsection{Space Weather}
% Relevancy of space weather to commercial airlines.
% Astronauts' time in space is decreasing.

The radiation environment in the Earth-Moon system has become a cause for concern over the last several years as comparisons between the levels of solar radiation during the solar cycle have been made. The increase in length of the solar minimum phase, and subsequent decrease in duration of the solar maximum, corresponds to record high cosmic ray activity.  A comparison between the number of days a 30 year old male could spend in interplanetary space in the 1990's versus 2014 revealed a substantial decrease from 1000 days in the 1990's to approximately 700 in 2014.   



Space radiation and galactic cosmic rays (GCRs) pose real health risks to astronauts and pilots in any field.  With the increase in cosmic ray flux   in the Earth Moon system, radiation exposure in Low Earth Orbit and the atmosphere must be monitored to ensure they do not exceed the occupational dose limits.  The ability to measure real-time exposure is arguably a crucial tool that can help mitigate increased health risks.  Additionally, Studying radiation types and dose levels near the nebulous border between space and Earth's atmosphere, and within the various zones of the atmosphere, facilitates preparation for current and future space missions; and aviation in general.  

%To monitor the radiation in the upper atmosphere and beyond, a USB TimePIX device could be effectively used as a low cost dosimeter.  TimePIX detectors have many applications in the realm of particle physics, specially with their small physical and power consumption footprint. Overall, TimePIX devices could be setup to create a network of devices to better understand the Earth space radiation environment and observe space weather.
%


\subsection{Success of SORA Relative to Others}
% Detail the results which match NASA's article about the disappearance of the plotzer maximum
% Mention the need for a low-cost dosimeter and how our configuration is reaching that goal. The only setback in the high-cost of the commercially unavailable MiniPIX.
% Mars environment
%\subsection{Future Work}
