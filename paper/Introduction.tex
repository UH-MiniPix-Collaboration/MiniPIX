\section{Introduction}
\label{Introduction}

During the HASP 2017\cite{hasp} flight the MiniPIX hybrid pixel detector\cite{minipix} was used to measure levels of cosmic radiation experienced by 
the SORA payload\cite{sora}. Presented in this paper are the methods and measurements used and recorded during the aforementioned HASP stratospheric balloon flight. 
Designed by CERN\cite{cern}, the MiniPIX utilizes a TimePIX\cite{timepix} silicon detector and relays the recorded data through its USB 2.0 interface provided by 
ADVACAM\cite{advacam}. The detector boarded a Raspberry PI 3 (RPI3) which ran the device in time over threshold (TOT) mode measuring deposited energy onto each 
pixel per frame throughout the flight. Frames were collected every 4 seconds with roughly a 6 second reset period. Data acquisitions began at power up prior to 
launch and ended at power down prior to freefall. The HASP 2017 flight launched from Fort Sumner, New Mexico on August 4th, 2017 at 14:04 UTC 
and ascended to a float altitude of about $\SI{31.5}{\kilo\meter}$ at 16:22 UTC which was maintained for about 10.5 hours. The payload drifted west for a total 
ground distance of \SI{580}{\kilo\meter} and was recovered just north of the Apache-Sitgreaves National Forest in Arizona.
