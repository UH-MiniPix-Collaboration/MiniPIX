\section{Introduction}
\label{Introduction}
The atmospheric radiation environments is a highly complex mix of primary and secondary particles originating from Galactic Cosmic Rays (GCR). In the earth’s atmosphere, primary GCR interact with the molecules in the upper atmosphere and produce cascades of secondary particles that increase in intensity until reaching a maximum between 16km and 18km known as the Regener-Pfotzer Maximum. After reaching peak ionization rates, the flux of secondary and primary particles then decreases steadily until reaching a minimum near the earths surface(cite). This peak in ionization in the atmosphere has implications to passengers and crew aboard high flying aircraft and astronauts, who will absorb a much higher biological dose than they would at the earths surface.  Crew and passengers aboard a standard passenger plane flying at a cruising altitude of approximately 13km are exposed to dose rates that are 40 times higher than they would experience on the earths surface(cite) and crew members aboard the International Space Station can be exposed to almost 100 times the typical dose rate at sea level(cite). Additionally, data from the CraTER cosmic ray telescope aboard NASA’s Lunar Reconnaissance Orbiter has shown that cosmic ray flux in the earth moon system has been increasing over the last 50 years and has just recently reached a historical maximum(cite). These factors seem to point towards the necessity of the development of a low cost and accurate device for measuring absorbed dose from cosmic radiation in-situ rather than relying on simulation or modeling for calculating maximum exposure. 

During the High Altitude Student Platform (HASP) 2017 and 2018 flights \cite{hasp}, the SORA payload carried a MiniPIX hybrid pixel detector interfaced with a Raspberry Pi Single Board Computer to test the feasibility of such a system for real-time measurements of absorbed dose from cosmic radiation. The MiniPIX was set to operate in Time over Threshhold mode with a bias voltage of 4 Kev, frames were collected at static 4 second intervals during the 2017 flight and between 1 and 4 second intervals for the 2018 flight. Data from the MiniPIX was analyzed in real time and counts and absorbed dose were downlinked in real time through the HASP downlink interface. 
The HASP 2017 flight launched from Fort Sumner, New Mexico on August 4th, 2017 at 14:04 UTC and ascended to a float altitude of approximately $\SI{31.5}{\kilo\meter}$ at 16:22 UTC; which was maintained for approximately \SI{10.5}{\hour}. On September 4th, 2018 at 14:03 UTC the HASP 2018 mission also launched from Fort Sumner, Mew Mexico.  The 2018 payload reached a stable float altitude of $\SI{37.2}{\kilo\meter}$ at 16:30 UTC and the total float duration was approximately \SI{9.0}{\hour}.
Data acquisition began at power up prior to launch and frames were collected every \SI{4}{\second}, using a bias voltage of $\SI{4}{\kilo\electronvolt}$, until the payload was powered down shortly before free-fall was initiated. The first payload drifted west for a total ground distance of \SI{580}{\kilo\meter} and was recovered just north of the Apache-Sitgreaves National Forest in Arizona. The 2018 payload terminated its flight and landed approximately \SI{96.6}{\kilo\meter} southwest of Mt Graham, Arizona after traveling a total distance of \SI{550}{\kilo\meter}.

