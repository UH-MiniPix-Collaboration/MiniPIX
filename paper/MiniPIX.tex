%% This is file `elsarticle-template-1-num.tex',
%%
%% Copyright 2009 Elsevier Ltd
%%
%% This file is part of the 'Elsarticle Bundle'.
%% ---------------------------------------------
%%
%% It may be distributed under the conditions of the LaTeX Project Public
%% License, either version 1.2 of this license or (at your option) any
%% later version.  The latest version of this license is in
%%    http://www.latex-project.org/lppl.txt
%% and version 1.2 or later is part of all distributions of LaTeX
%% version 1999/12/01 or later.
%%
%% Template article for Elsevier's document class `elsarticle'
%% with numbered style bibliographic references
%%
%% $Id: elsarticle-template-1-num.tex 149 2009-10-08 05:01:15Z rishi $
%% $URL: http://lenova.river-valley.com/svn/elsbst/trunk/elsarticle-template-1-num.tex $

\documentclass[final,1p, times, twocolumn]{elsarticle}

%% Use the option review to obtain double line spacing
%% \documentclass[preprint,review,12pt]{elsarticle}

%% Use the options 1p,twocolumn; 3p; 3p,twocolumn; 5p; or 5p,twocolumn
%% for a journal layout:
%% \documentclass[final,1p,times]{elsarticle}
%% \documentclass[final,1p,times,twocolumn]{elsarticle}
%% \documentclass[final,3p,times]{elsarticle}
%% \documentclass[final,3p,times,twocolumn]{elsarticle}
%% \documentclass[final,5p,times]{elsarticle}
%% \documentclass[final,5p,times,twocolumn]{elsarticle}

%% The graphicx package provides the includegraphics command.
\usepackage{graphicx}
%% The amssymb package provides various useful mathematical symbols
\usepackage{amssymb}
%% The amsthm package provides extended theorem environments
%% \usepackage{amsthm}

%% The lineno packages adds line numbers. Start line numbering with
%% \begin{linenumbers}, end it with \end{linenumbers}. Or switch it on
%% for the whole article with \linenumbers after \end{frontmatter}.
\usepackage{lineno}
\usepackage{siunitx}
%% natbib.sty is loaded by default. However, natbib options can be
%% provided with \biboptions{...} command. Following options are
%% valid:

\usepackage{caption}
\usepackage{subcaption}
\usepackage{hyperref}

%%   round  -  round parentheses are used (default)
%%   square -  square brackets are used   [option]
%%   curly  -  curly braces are used      {option}
%%   angle  -  angle brackets are used    <option>
%%   semicolon  -  multiple citations separated by semi-colon
%%   colon  - same as semicolon, an earlier confusion
%%   comma  -  separated by comma
%%   numbers-  selects numerical citations
%%   super  -  numerical citations as superscripts
%%   sort   -  sorts multiple citations according to order in ref. list
%%   sort&compress   -  like sort, but also compresses numerical citations
%%   compress - compresses without sorting
%%
%% \biboptions{comma,round}

% \biboptions{}

%% \usepackage{authblk}
%% authors package


\graphicspath{{../figures/}}
\journal{Nuclear Instruments and Methods in Physics Research}

\begin{document}

\begin{frontmatter}

%% Title, authors and addresses

\title{MiniPIX Cosmic Ray Tracking and Radiation Dosimetry During HASP Stratospheric Balloon Flight}
%% title is a work in progress...

%% use the tnoteref command within \title for footnotes;
%% use the tnotetext command for the associated footnote;
%% use the fnref command within \author or \address for footnotes;
%% use the fntext command for the associated footnote;
%% use the corref command within \author for corresponding author footnotes;
%% use the cortext command for the associated footnote;
%% use the ead command for the email address,
%% and the form \ead[url] for the home page:
%%
%% \title{Title\tnoteref{label1}}
%% \tnotetext[label1]{}

% To avoid any authorship arguments I would suggest we order authors in alphabetical order by last name. Just a thought.
\author{S.~A.~Garcia~Morelos\corref{cor1}\fnref{label2}}
\author{F.~Brooks\corref{cor2}\fnref{label3}}
\author{S.~Oliver\corref{cor3}\fnref{label4}}
\author{A.~Walker\corref{cor4}\fnref{label5}}
\author{K.~D.~Portillo\corref{cor5}\fnref{label6}}
\author{R.~B.~Masek\corref{cor6}\fnref{label7}}
\author{S.~George\corref{cor7}\fnref{label8}}
\author{D.~Pattison\corref{cor8}\fnref{label9}}
\author{A.~L.~Renshaw\corref{cor9}\fnref{label10}}

%% \ead{email address}
%% \ead[url]{home page}
%% \fntext[label2]{}
%% \cortext[cor1]{}
%% \address{Address\fnref{label3}}
%% \fntext[label3]{}


%% use optional labels to link authors explicitly to addresses:
%% \author[label1,label2]{<author name>}
%% \address[label2]{<address>}
%% \address[label3]{<address>}

\address[label2,label3,label4,label5,label6,label7,label8,label9,label10]{Department of Physics, University of Houston, Houston, TX 77204, USA}
%\address[label3]{Department of Physics, University of Houston, Houston, TX 77204, USA}
%\address[label4]{Department of Physics, University of Houston, Houston, TX 77204, USA}
%\address[label5]{Department of Physics, University of Houston, Houston, TX 77204, USA}
%\address[label6]{Department of Physics, University of Houston, Houston, TX 77204, USA}
%\address[label7]{Department of Physics, University of Houston, Houston, TX 77204, USA}
%\address[label8]{Department of Physics, University of Houston, Houston, TX 77204, USA}
%\address[label9]{Department of Physics, University of Houston, Houston, TX 77204, USA}
%\address[label10]{Department of Physics, University of Houston, Houston, TX 77204, USA}

%% \author{John Smith}
%% \address{California, United States}
%% \newcommand{\Houston}{Department of Physics, University of Houston, Houston, TX 77204, USA}
%--- Add other authors in the order they should appear

\begin{abstract}

The results of the SORA payload during the 2017 HASP stratospheric ballooon flight are presented. Let's write the paper first then finish the abstract.
\end{abstract}

%Required Structure: https://www.elsevier.com/journals/nuclear-instruments-and-methods-in-physics-research-section-a-accelerators-spectrometers-detectors-and-associated-equipment/0168-9002/guide-for-authors

%optional: Highlights
%Highlights are a short collection of bullet points that convey the core findings of the article. Highlights are optional and should be submitted in a separate editable file in the online submission system. Please use 'Highlights' in the file name and include 3 to 5 bullet points (maximum 85 characters, including spaces, per bullet point). You can view example Highlights on our information site.

\begin{keyword}
MiniPIX \sep TimePIX \sep HASP \sep SORA \sep Cosmic Radiation \sep Stratospheric Balloon \sep Dosimetry
%% keywords here, in the form: keyword \sep keyword

%% MSC codes here, in the form: \MSC code \sep code
%% or \MSC[2008] code \sep code (2000 is the default)

\end{keyword}

%Abbreviations! we need to define them here.  Define abbreviations that are not standard in this field in a footnote to be placed on the first page of the article. Such abbreviations that are unavoidable in the abstract must be defined at their first mention there, as well as in the footnote. Ensure consistency of abbreviations throughout the article.
%%Abbreviations
%%SORA (Stratospheric Organisms and Radiation Analyzer
%%UH (University of Houston)
%%NASA
%%MiniPIX
%%HASP
%%LSU
%%CERN
%%Mega (arduino mega)
%%Pi (Raspberry Pi)
%%flight pc (fligth computer)
%%km (kilometers)
%%m (meters)

\end{frontmatter}

%%
%% Start line numbering here if you want
%%
\linenumbers


%% The Appendices part is started with the command \appendix;
%% appendix sections are then done as normal sections
%% \appendix

%% \section{}
%% \label{}

%% References
%%
%% Following citation commands can be used in the body text:
%% Usage of \cite is as follows:
%%   \cite{key}          ==>>  [#]
%%   \cite[chap. 2]{key} ==>>  [#, chap. 2]
%%   \citet{key}         ==>>  Author [#]

%---Introduction
\section{Introduction}
\label{Introduction}

During the HASP 2017 and 2018 flights \cite{hasp} , the MiniPIX hybrid pixel detector\cite{minipix} was used to track cosmic rays in the stratosphere. 
%
This paper presents an overview of the design of the radiation measurement  instrumentation on the SORA payload in conjunction with the data acquired  from both flights.
%
The MiniPIX utilizes a TimePIX\cite{timepix} silicon detector designed by CERN\cite{cern} and a USB 2.0 readout interface provided by ADVACAM\cite{advacam}. 
%
The detector boarded a Raspberry PI 3 (RPI3) ;which ran the device in time over threshold (TOT) mode measuring deposited energy onto each pixel per frame throughout the flight. 
%
Frames were collected every \SI{4}{\second} using a bias voltage of $\SI{4}{\kilo\electronvolt}$ . 
%include settings for the second flight too.
%
Data acquisition began at power up prior to launch and continued until the payload was powered down shortly before free-fall. 
%
The HASP frame test2017 flight launched from Fort Sumner, New Mexico on August 4th, 2017 at 14:04 UTC and ascended to a float altitude of about $\SI{31.5}{\kilo\meter}$ at 16:22 UTC which was maintained for about \SI{10.5}{\hour}.
%As for the HASP 2018 flight, it launch from the same Fort sumner location.  It took off on September 4th, 2018 at 14:03 UTC.  It reached a stable float altitude of $\SI{37.2}{\kilo\meter}$ at 16:30 UTC.  Total float duration was about \SI{9.0}{\hour}.
%
 The first payload drifted west for a total ground distance of \SI{580}{\kilo\meter} and was recovered just north of the Apache-Sitgreaves National Forest in Arizona. 
 %
 A year later, the second payload terminated its flight and then landed shortly about \SI{96.6}{\kilo\meter} southwest of Mt Graham, Arizona after traveling a total of \SI{550}{\kilo\meter}.
 %
 HASP (High Altitude Student Platform) provided both opportunities for the SORA payloads to take flight onboard high altitude balloons. The HASP flight program is further supported by NASA BPO (Balloon Progam Office) and LaSPACE (Louisiana Space Consortium).


%---Background
% I suggest we discuss briefly Timepix detectors use on the ISS and other places for space radiation dosimetry and particle identification. We can also talk a bit about our motivations of studying cosmic rays on commercial airline flights.

\section{Background}
\label{Background}

TimePIX detectors have many applications in the realm of Particle Physics. 
%
Numerous balloon flights have included the use of TimePIX devices for the purposes of particle imaging and radiation dosimetry in unusual and unfamiliar environments, such as the stratosphere \cite{bexus}. 
%
The International Space Station (ISS) uses TimePIX devices to evalute the astronauts' exposure to ionizing radiation fields \cite{timepixiss}.
%below is an actual space

The TimePIX device used in this flight was a MiniPIX, which is a silicon-based hybrid pixel detector built by ADVACAM \cite{advacam}. 
%
The sensor surface offers a resolution of 256x256 pixels.
%
The device can be operated under one of three modes: time-of-arrival (TOA) or time-over-threshold (TOT), or single particle counting. 
%
When a particle ionizes with the silicon chip, the deposited charge is depleted by a bias voltage in a process known as Carrier generation and recombination. \textbf{<<Verify the previous statement>>}
%below is an actual space

\begin{figure}[h]
    \centering
    \includegraphics[width=0.25\textwidth]{minipix_silicon.pdf}
    \caption{MiniPIX silicon concept design}
    \label{fig:minipix_silicon}
\end{figure}
%

\textbf{<<Include silicon-chip figure?>>}
%figure name: minipix_silicon.pdf


%---Experiment Design
\section{Methods}
\label{Methods}
% This should be fairly brief, the system is not particularly complex.
For the SORA flights, a MiniPIX with a USB interface was coupled with a low cost open ARM based computer called Raspberry Pi 3.  Both flights utilized a Raspberry Pi 3 to communicate and handle all operations with the MiniPIX.  This system allowed for remote operation and on-board analyzation of all data.  The SORA flights tested the feasibility of these low-cost systems to fly further upper atmospheric and near-space missions.  

The PyPixet library, developed by Daniel Turcek (CITE ME), controlled the MiniPIX device used in both SORA flights.  The 2017 flight was the first testbed for MiniPIX setup, testing the limits of the hardware and software setup.  The 2018 flight was more robust and applied many improvements based on challenges faced during the initial flight.  Important changes included support for an arbitrary number of MiniPIX detectors to record data simultaneously.  Furthermore, the 2018 flight software allowed for direct serial uplink and downlink. This supported observation of real-time data, adjusting the MiniPIX's shutter rate, and controlling the acquisition settings.

\subsection{Configuration and Calibration}
% Device parameters, threshold, bias voltage, shutter time etc.
The appropriate calibration of the MiniPIX detector was applied at The University of Houston by Dr. Stuart P. George, a collaborator within the Medipix Collaboration. The source calibration was applied using the 60 keV $^{241}$Am decay line, Sn Fluorescence and $^{55}$Fe gamma rays. The Timepix hybrid pixel detector
consists of 65,536 silicon p-n diodes, with each containing its own individual processing circuit. The response
of each pixel can never be identical, thus a calibration must be performed for each individual pixel. 

The data was recorded using the Pixet Pro software provided by ADVACAM ~\ref{}. Dr. George calibrated
the pixel energy threshold from DAC counts to energy ~\ref{}. The threshold was set at 4 keV, just above
the noise level of the detector. The set threshold energy of the Timepix chip determines what energies of
particles are allowed to be measured by the detector. Energy measurements in the detector are accounted
by measuring the charge collected in each individual pixel.

\subsection{System Design}
% RPI interface to MiniPIX, heatsink design etc.
Additive manufacturing was utilized to create a custom made case for the MiniPIX device.  As seen in Figure ~\ref{fig:minipix_case}, the MiniPIX case is in white, while the heat sink setup is in blue-gray color.  In the early vacuum tests, it was found that the MiniPIX would require a passive cooling method to keep the device at operational levels.  At the same time, the ABS plastic enclosure acted as insulation, maintaining the MiniPIX temperatures in operational ranges.  The MiniPIX device was mated to the heatsink via a bracket (not shown in the figure ~\ref{fig:minipix_case}) and with thermal adhesive.
\begin{figure}[h]
    \centering
    \includegraphics[width=0.25\textwidth]{Minipix_case_assembly.pdf}
    \caption{MiniPIX Case Assembly}
    \label{fig:minipix_case}
\end{figure}
The MiniPIX case allowed for a USB cable to be routed through the enclosure to interface directly to the Raspberry Pi.  This allowed the MiniPIX device to be modular and be place in different configurations for both flights.  The Raspberry Pi was placed in a separate location within the payloads near the power supply.  Overall, the payload was modular and accessible.





%---Data Analysis
\section{Results}
\label{Results}
%\section{Data Analysis}
%\label{Data Analysis}

\subsection{Vertical Radiation Profile}
%Notes on how to take the results...
%What is being presented: Five plots total, the flight profile for each year, and cumulative dose for each flight.

\begin{figure}[H]
\centering
\includegraphics[width=\textwidth]{count_rate_altitude_2017.pdf}%REMINDER:Ask Andrew if he can make the text on this plot bigger, and lets make a standard to use PDF only for all images.
\caption{Count Rate and Altitude vs Time for 2017 flight}
\label{fig:ratealttime_2017}
\end{figure}
%First:
%Present and characterize the first two plots - the data of the flight with counts vs altitude vs time.  Here we can talk about each flight and data shown.  Make sure to mention how the count rate changed as time and altitude changed.  We can talk about the flight time, the flight altitude variance, the coasting altitude, and how it all compares to the count rate.  We can be as descriptive as we can.
%mention the position of the pfotzer-regener maximum for both years.  Talk about discrepancies as the data shows.  Speculate in discussion later.

Both HASP balloons containing the SORA payloads were launched separately on September 4th on 2017 and 2018 respectively.  The MiniPIX was set to operate in Time over Threshold mode with a bias voltage of 4 Kev.  Frames were collected at static 4 second intervals during the 2017 flight and between 1 and 4 second intervals for the 2018 flight.  The complete flight altitude profiles with total count rates are shown in Figure~\ref{fig:ratealttime_2017} for 2017 and Figure~\ref{fig:ratealttime_2018} for 2018.

As shown in Figure~\ref{fig:ratealttime_2017} and Figure~\ref{fig:ratealttime_2018}, both flights have very similar flight profiles.  It took each flight about 2 hours and 30 minutes to ascend to a stable float altitude.  It is important to mention that the 2017 flight reach and maintained a float altitude of approximately $\SI{31.5}{\kilo\meter}$.  In contrast, the 2018 flight kept a float altitude of $\SI{37.2}{\kilo\meter}$.  The rate of decent was slow and steady for both flights.  As such, data was collected continuously throughout the whole flights.

With both flights reaching altitudes beyond  $\SI{25.0}{\kilo\meter}$, the Pfotzer-Regener maximum was observed very clearly.  In both Figure~\ref{fig:ratealttime_2017} and Figure~\ref{fig:ratealttime_2018}, the 

\begin{figure}[H]
\centering
\includegraphics[width=\textwidth]{count_rate_altitude_2018.pdf}
\caption{PLACEHOLDER Count Rate and Altitude vs Time for 2018 flight}
\label{fig:ratealttime_2018}
\end{figure}

\begin{figure}[H]
\centering
\begin{subfigure}{.5\textwidth}
  \centering
  \includegraphics[scale=.45]{dva_stderr.pdf}
  \caption{Dose rate in silicon vs. altitude.}
  \label{fig:sub1}
\end{subfigure}%
\begin{subfigure}{.5\textwidth}
  \centering
  \includegraphics[scale=.45]{cva_stderr.pdf}
  \caption{Detector Hits vs. altitude.}
  \label{fig:sub2}
\end{subfigure}
\caption{Figure ~\ref{fig:sub1} shows the absorbed dose rate per hour as a function of altitude from the MiniPIX.  Figure ~\ref{fig:sub2} shows the counts per minute as a function of altitude again from the MiniPIX data.}
\label{fig:test}
\end{figure}
%Next, go into the absorbed rate vs altitude plot.  Mention how the LET varies for materials such as silicon and muscle.  Go into the pfotzer-regener maximum again, compare.  Compare this to the accompanying figure, Detector Hits vs Altitude.  There is a slight discrepancy in the flights, this may be useful to mention.  All error bars are 1 sigma standard deviation.

\begin{figure}[H]
\centering
\includegraphics[scale=.75]{ctva-cropped.pdf}
\caption{Cluster Type Counts vs. altitude.}
\end{figure}
%Finally, the last figure Cluster Type Counts vs Altitude.  This is useful due to the MiniPIX being able to analyze indivudual track lenghts.  From here, these tracks can be characterized into different and indidivual categories.  This is useful for LET calculations and overall more precise for dosage calculations.  It may also help with particle identification.  Notice how the heavier tracks and medium blobs are all in the low counts yet they still vary somewhat with altitude (hard to see).

\begin{figure}[H]
\centering
\includegraphics[scale=.35]{tracks.png}
\caption{Frame collected at float.}
\end{figure}



%---Discussion
\section{Discussion}
\label{Discussion}
\subsection{System Performance}
\subsection{Results}
% Discuss dose and particle count measurements as they relate to classical theory


%---Conclusion
\section{Conclusion}
\label{Conclusion}
% Overall, the device performed adequately and is potentially suitable for future applications in other areas
% Discuss future applications of the MiniPIX for compact, cheap and accurate(?) dosimeters on commercial airline flights


%---Acknowledgements
%Acknowledgements
%Collate acknowledgements in a separate section at the end of the article before the references and do not, therefore, include them on the title page, as a footnote to the title or otherwise. List here those individuals who provided help during the research (e.g., providing help during the assembly process, language help, writing assistance or proof reading the article, etc.).

\section{Acknowledgements}
\label{Acknowledgements}
This will be a list populated by all the people that helped contribute to the project at one point or the other.  This includes those that helped us build briefly during the summer, and also those that helped review our paper.


%---Funding Sources
\section{Funding Sources}
\label{funding}
This work was supported by the University of Houston STEM Center.
%%Role of the funding source
%You are requested to identify who provided financial support for the conduct of the research and/or preparation of the article and to briefly describe the role of the sponsor(s), if any, in study design; in the collection, analysis and interpretation of data; in the writing of the report; and in the decision to submit the article for publication. If the funding source(s) had no such involvement then this should be stated.
%%%%
%Formatting of funding sources
%List funding sources in this standard way to facilitate compliance to funder's requirements:
%
%Funding: This work was supported by the National Institutes of Health [grant numbers xxxx, yyyy]; the Bill & Melinda Gates Foundation, Seattle, WA [grant number zzzz]; and the United States Institutes of Peace [grant number aaaa].
%
%It is not necessary to include detailed descriptions on the program or type of grants and awards. When funding is from a block grant or other resources available to a university, college, or other research institution, submit the name of the institute or organization that provided the funding.
%
%If no funding has been provided for the research, please include the following sentence:
%
%This research did not receive any specific grant from funding agencies in the public, commercial, or not-for-profit sectors.


%---Appendix
%% The Appendices part is started with the command \appendix;
%% appendix sections are then done as normal sections
%% \appendix

%% \section{}
%% \label{}


%---Bibliography
%% References
%%
%% Following citation commands can be used in the body text:
%% Usage of \cite is as follows:
%%   \cite{key}          ==>>  [#]
%%   \cite[chap. 2]{key} ==>>  [#, chap. 2]
%%   \citet{key}         ==>>  Author [#]

%% References with bibTeX database:
%%%%%%%%
%%%%%%%%\bibliographystyle{model1-num-names}
%%%%%%%%\bibliography{sample.bib}

%% Authors are advised to submit their bibtex database files. They are
%% requested to list a bibtex style file in the manuscript if they do
%% not want to use model1-num-names.bst.

%% References without bibTeX database:

\begin{thebibliography}{00}

%\bibitem{SORA}S.A. Garcia Morelos, F.Brooks, S.Oliver, A.Walker, K.D. Portillo, R.B. Masek, D.Mroczek, D.Pena, J.Juarez, A.Cruz, D. Henandez, S.George, D. Pattison, A.L.Renshaw. \textit{Scientific Report for the UH Team.} SORA 2017 Mission Webpage. \textit{http://laspace.lsu.edu/hasp/groups/Payload.php?py=2017&pn=10}.

\bibitem{minipix}MiniPIX - Miniaturized Portable USB Photon Counting Camera. (n.d.). Retrieved February 02, 2017, from \textit{http://advacam.com/camera/minipix}.

\bibitem{timepixiss} Stoffle, N., Pinsky, L. et al. Timepix-based radiation environment monitor measurements aboard the International Space Station. Nucl. Instrum. Methods Phys.
Res., A, 782 (2015):143.

\bibitem{aircrewexposure} B.J. Lewis, M.J. McCall, A.R. Green, L.G.I. Bennett, M. Pierre, U.J. Schrewe, K. O'Brien, E. Felsberger; Aircrew Exposure from Cosmic Radiation on Commercial Airline Routes, Radiation Protection Dosimetry, Volume 93, Issue 4, 1 February 2001, Pages 293–314, \url{https://doi-org.ezproxy.lib.uh.edu/10.1093/oxfordjournals.rpd.a006442}
\bibitem{timepixdosimetry} Carlos Granja, Stanislav Pospisil,
Quantum dosimetry and online visualization of X-ray and charged particle radiation in commercial aircraft at operational flight altitudes with the pixel detector Timepix,
Advances in Space Research,
Volume 54, Issue 2,
2014,
Pages 241-251,
ISSN 0273-1177,
https://doi.org/10.1016/j.asr.2014.04.006.
(http://www.sciencedirect.com/science/article/pii/S0273117714002208)
Keywords: Radiation detection; Radiation exposure at aviation altitudes; Dose equivalent rate; Semiconductor pixel detector; Quantum dosimetry

\bibitem{advacam}
  ADVACAM at \url{http://www.advacam.com}

\bibitem{medipix}
  Medipix collaboration at \url{https://medipix.web.cern.ch/}.
  
\bibitem{stuartthesis} 22
  George, S., \textit{Dosimetric Applications of Hybrid Pixel Detectors}, University of Wollongong, Australia, 2015.

\bibitem{mpdatasheet}
  ADVACAM, \textit{MINIPIX Version 1.0 Datasheet}, Retrieved from \url{http://www.widepix.cz/files/datasheets/MiniPIX\%20v1.0\%20Datasheet.pdf}.

\bibitem{mpjakubek}
  Jan Jakubek, \textit{Precise energy calibration of pixel detector working in time-over-threshold mode} Institute of Experimental and Applied Physics, Czech Technical University in Prague, Czech Republic, 2011.

\bibitem{bexus}
Urbar, J., Scheirich, J., Jakubek, J., 2011. Medipix/Timepix cosmic ray tracking on BEXUS stratospheric balloonflights. Nucl. Instrum. Methods A 633, S206-209.
  
\bibitem{hasp}
http://laspace.lsu.edu/hasp/
\end{thebibliography}




\end{document}

%%
%% End of file `elsarticle-template-1-num.tex'.
              
\grid
\grid
