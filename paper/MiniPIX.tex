%% This is file `elsarticle-template-1-num.tex',
%%
%% Copyright 2009 Elsevier Ltd
%%
%% This file is part of the 'Elsarticle Bundle'.
%% ---------------------------------------------
%%
%% It may be distributed under the conditions of the LaTeX Project Public
%% License, either version 1.2 of this license or (at your option) any
%% later version.  The latest version of this license is in
%%    http://www.latex-project.org/lppl.txt
%% and version 1.2 or later is part of all distributions of LaTeX
%% version 1999/12/01 or later.
%%
%% Template article for Elsevier's document class `elsarticle'
%% with numbered style bibliographic references
%%
%% $Id: elsarticle-template-1-num.tex 149 2009-10-08 05:01:15Z rishi $
%% $URL: http://lenova.river-valley.com/svn/elsbst/trunk/elsarticle-template-1-num.tex $
%%
\documentclass[preprint,12pt]{elsarticle}

%% Use the option review to obtain double line spacing
%% \documentclass[preprint,review,12pt]{elsarticle}

%% Use the options 1p,twocolumn; 3p; 3p,twocolumn; 5p; or 5p,twocolumn
%% for a journal layout:
%% \documentclass[final,1p,times]{elsarticle}
%% \documentclass[final,1p,times,twocolumn]{elsarticle}
%% \documentclass[final,3p,times]{elsarticle}
%% \documentclass[final,3p,times,twocolumn]{elsarticle}
%% \documentclass[final,5p,times]{elsarticle}
%% \documentclass[final,5p,times,twocolumn]{elsarticle}

%% The graphicx package provides the includegraphics command.
\usepackage{graphicx}
%% The amssymb package provides various useful mathematical symbols
\usepackage{amssymb}
%% The amsthm package provides extended theorem environments
%% \usepackage{amsthm}

%% The lineno packages adds line numbers. Start line numbering with
%% \begin{linenumbers}, end it with \end{linenumbers}. Or switch it on
%% for the whole article with \linenumbers after \end{frontmatter}.
\usepackage{lineno}

%% natbib.sty is loaded by default. However, natbib options can be
%% provided with \biboptions{...} command. Following options are
%% valid:

%%   round  -  round parentheses are used (default)
%%   square -  square brackets are used   [option]
%%   curly  -  curly braces are used      {option}
%%   angle  -  angle brackets are used    <option>
%%   semicolon  -  multiple citations separated by semi-colon
%%   colon  - same as semicolon, an earlier confusion
%%   comma  -  separated by comma
%%   numbers-  selects numerical citations
%%   super  -  numerical citations as superscripts
%%   sort   -  sorts multiple citations according to order in ref. list
%%   sort&compress   -  like sort, but also compresses numerical citations
%%   compress - compresses without sorting
%%
%% \biboptions{comma,round}

% \biboptions{}

\journal{Journal Name}

\begin{document}

\begin{frontmatter}

%% Title, authors and addresses

\title{Unnecessarily Complicated Research Title}

%% use the tnoteref command within \title for footnotes;
%% use the tnotetext command for the associated footnote;
%% use the fnref command within \author or \address for footnotes;
%% use the fntext command for the associated footnote;
%% use the corref command within \author for corresponding author footnotes;
%% use the cortext command for the associated footnote;
%% use the ead command for the email address,
%% and the form \ead[url] for the home page:
%%
%% \title{Title\tnoteref{label1}}
%% \tnotetext[label1]{}
%% \author{Name\corref{cor1}\fnref{label2}}
%% \ead{email address}
%% \ead[url]{home page}
%% \fntext[label2]{}
%% \cortext[cor1]{}
%% \address{Address\fnref{label3}}
%% \fntext[label3]{}


%% use optional labels to link authors explicitly to addresses:
%% \author[label1,label2]{<author name>}
%% \address[label1]{<address>}
%% \address[label2]{<address>}

\author{John Smith}

\address{California, United States}

\begin{abstract}
%% Text of abstract
Suspendisse potenti. Suspendisse quis sem elit, et mattis nisl. Phasellus consequat erat eu velit rhoncus non pharetra neque auctor. Phasellus eu lacus quam. Ut ipsum dolor, euismod aliquam congue sed, lobortis et orci. Mauris eget velit id arcu ultricies auctor in eget dolor. Pellentesque suscipit adipiscing sem, imperdiet laoreet dolor elementum ut. Mauris condimentum est sed velit lacinia placerat. Vestibulum ante ipsum primis in faucibus orci luctus et ultrices posuere cubilia Curae; Nullam diam metus, pharetra vitae euismod sed, placerat ultrices eros. Aliquam tincidunt dapibus venenatis. In interdum tellus nec justo accumsan aliquam. Nulla sit amet massa augue.
\end{abstract}

\begin{keyword}
Science \sep Publication \sep Complicated
%% keywords here, in the form: keyword \sep keyword

%% MSC codes here, in the form: \MSC code \sep code
%% or \MSC[2008] code \sep code (2000 is the default)

\end{keyword}

\end{frontmatter}

%%
%% Start line numbering here if you want
%%
\linenumbers

%% main text
\section{The First Section}
\label{S:1}

Maecenas \cite{Smith:2012qr} fermentum \cite{Smith:2013jd} urna ac sapien tincidunt lobortis. Nunc feugiat faucibus varius. Ut sed purus nunc. Ut eget eros quis lectus mollis pharetra ut in tellus. Pellentesque ultricies velit sed orci pharetra et fermentum lacus imperdiet. Class aptent taciti sociosqu ad litora torquent per conubia nostra, per inceptos himenaeos. Suspendisse commodo ultrices mauris, condimentum hendrerit lorem condimentum et. Pellentesque urna augue, semper et rutrum ac, consequat id quam. Proin lacinia aliquet justo, ut suscipit massa commodo sit amet. Proin vehicula nibh nec mauris tempor interdum. Donec orci ante, tempor a viverra vel, volutpat sed orci.

Pellentesque habitant morbi tristique senectus et netus et malesuada fames ac turpis egestas. Pellentesque quis interdum velit. Nulla tincidunt sem quis nisi molestie nec hendrerit nulla interdum. Nunc at lectus at neque dapibus dapibus sit amet in massa. Nam ut nisl in diam consectetur dignissim. Sed lacinia diam id nunc suscipit vitae semper lorem semper. In vehicula velit at tortor fringilla elementum aliquam erat blandit. Donec pretium libero et neque vehicula blandit. Curabitur consequat interdum sem at ultrices. Sed at tincidunt metus. Etiam vulputate, lacus eget fermentum posuere, ante mi dignissim augue, et ultrices felis tortor sed nisl.

\begin{itemize}
\item Bullet point one
\item Bullet point two
\end{itemize}

\begin{enumerate}
\item Numbered list item one
\item Numbered list item two
\end{enumerate}

\subsection{Subsection One}

Quisque elit ipsum, porttitor et imperdiet in, facilisis ac diam. Nunc facilisis interdum felis eget tincidunt. In condimentum fermentum leo, non consequat leo imperdiet pharetra. Fusce ac massa ipsum, vel convallis diam. Quisque eget turpis felis. Curabitur posuere, risus eu placerat porttitor, magna metus mollis ipsum, eu volutpat nisl erat ac justo. Nullam semper, mi at iaculis viverra, nunc velit iaculis nunc, eu tempor ligula eros in nulla. Aenean dapibus eleifend convallis. Cras ut libero tellus. Integer mollis eros eget risus malesuada fringilla mattis leo facilisis. Etiam interdum turpis eget odio ultricies sed convallis magna accumsan. Morbi in leo a mauris sollicitudin molestie at non nisl.

\begin{table}[h]
\centering
\begin{tabular}{l l l}
\hline
\textbf{Treatments} & \textbf{Response 1} & \textbf{Response 2}\\
\hline
Treatment 1 & 0.0003262 & 0.562 \\
Treatment 2 & 0.0015681 & 0.910 \\
Treatment 3 & 0.0009271 & 0.296 \\
\hline
\end{tabular}
\caption{Table caption}
\end{table}

\subsection{Subsection Two}

Donec eget ligula venenatis est posuere eleifend in sit amet diam. Vestibulum sollicitudin mauris ac augue blandit ultricies. Nulla facilisi. Etiam ut turpis nunc. Praesent leo orci, tincidunt vitae feugiat eu, feugiat a massa. Duis mauris ipsum, tempor vel condimentum nec, suscipit non mi. Fusce quis urna dictum felis posuere sagittis ac sit amet erat. In in ultrices lectus. Nulla vitae ipsum lectus, a gravida erat. Etiam quam nisl, blandit ut porta in, accumsan a nibh. Phasellus sodales euismod dolor sit amet elementum. Phasellus varius placerat erat, nec gravida libero pellentesque id. Fusce nisi ante, euismod nec cursus at, suscipit a enim. Nulla facilisi.

\begin{figure}[h]
\centering\includegraphics[width=0.4\linewidth]{placeholder}
\caption{Figure caption}
\end{figure}

Integer risus dui, condimentum et gravida vitae, adipiscing et enim. Aliquam erat volutpat. Pellentesque diam sapien, egestas eget gravida ut, tempor eu nulla. Vestibulum mollis pretium lacus eget venenatis. Fusce gravida nisl quis est molestie eu luctus ipsum pretium. Maecenas non eros lorem, vel adipiscing odio. Etiam dolor risus, mattis in pellentesque id, pellentesque eu nibh. Mauris nec ante at orci ultricies placerat ac non massa. Aenean imperdiet, ante eu sollicitudin vestibulum, dolor felis dapibus arcu, sit amet fermentum urna nibh sit amet mauris. Suspendisse adipiscing mollis dolor quis lobortis.

\begin{equation}
\label{eq:emc}
e = mc^2
\end{equation}

\section{The Second Section}
\label{S:2}

Reference to Section \ref{S:1}. Etiam congue sollicitudin diam non porttitor. Etiam turpis nulla, auctor a pretium non, luctus quis ipsum. Fusce pretium gravida libero non accumsan. Donec eget augue ut nulla placerat hendrerit ac ut mi. Phasellus euismod ornare mollis. Proin tempus fringilla ultricies. Donec pretium feugiat libero quis convallis. Nam interdum ante sed magna congue eu semper tellus sagittis. Curabitur eu augue elit.

Aenean eleifend purus et massa consequat facilisis. Etiam volutpat placerat dignissim. Ut nec nibh nulla. Aliquam erat volutpat. Nam at massa velit, eu malesuada augue. Maecenas sit amet nunc mauris. Maecenas eu ligula quis turpis molestie elementum nec at est. Sed adipiscing neque ac sapien viverra sit amet vestibulum arcu rhoncus.

Vivamus pharetra nibh in orci euismod congue. Pellentesque habitant morbi tristique senectus et netus et malesuada fames ac turpis egestas. Quisque lacus diam, congue vel laoreet id, iaculis eu sapien. In id risus ac leo pellentesque pellentesque et in dui. Etiam tincidunt quam ut ante vestibulum ultricies. Nam at rutrum lectus. Aenean non justo tortor, nec mattis justo. Aliquam erat volutpat. Nullam ac viverra augue. In tempus venenatis nibh quis semper. Maecenas ac nisl eu ligula dictum lobortis. Sed lacus ante, tempor eu dictum eu, accumsan in velit. Integer accumsan convallis porttitor. Maecenas pretium tincidunt metus sit amet gravida. Maecenas pretium blandit felis, ac interdum ante semper sed.

In auctor ultrices elit, vel feugiat ligula aliquam sed. Curabitur aliquam elit sed dui rhoncus consectetur. Cras elit ipsum, lobortis a tempor at, viverra vitae mi. Cras sed urna sed eros bibendum faucibus. Morbi vel leo orci, vel faucibus orci. Vivamus urna nisl, sodales vitae posuere in, tempus vel tellus. Donec magna est, luctus non commodo sit amet, placerat et enim.

%% The Appendices part is started with the command \appendix;
%% appendix sections are then done as normal sections
%% \appendix

%% \section{}
%% \label{}

%% References
%%
%% Following citation commands can be used in the body text:
%% Usage of \cite is as follows:
%%   \cite{key}          ==>>  [#]
%%   \cite[chap. 2]{key} ==>>  [#, chap. 2]
%%   \citet{key}         ==>>  Author [#]

%% References with bibTeX database:

\bibliographystyle{model1-num-names}
\bibliography{sample.bib}

%% Authors are advised to submit their bibtex database files. They are
%% requested to list a bibtex style file in the manuscript if they do
%% not want to use model1-num-names.bst.

%% References without bibTeX database:

% \begin{thebibliography}{00}

%% \bibitem must have the following form:
%%   \bibitem{key}...
%%

% \bibitem{}

% \end{thebibliography}


\end{document}

%%
%% End of file `elsarticle-template-1-num.tex'.
              
