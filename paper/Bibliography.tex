%% References
%%
%% Following citation commands can be used in the body text:
%% Usage of \cite is as follows:
%%   \cite{key}          ==>>  [#]
%%   \cite[chap. 2]{key} ==>>  [#, chap. 2]
%%   \citet{key}         ==>>  Author [#]

%% References with bibTeX database:
%%%%%%%%
%%%%%%%%\bibliographystyle{model1-num-names}
%%%%%%%%\bibliography{sample.bib}

%% Authors are advised to submit their bibtex database files. They are
%% requested to list a bibtex style file in the manuscript if they do
%% not want to use model1-num-names.bst.

%% References without bibTeX database:


\begin{thebibliography}{00}

\bibitem{hess}
  J.W.W.F, \textit{The Electrical Conductivity of the Atomsphere and its Causes}, Nature 123, 155-156, (1929).

\bibitem{abe}
  Abe, K \& Fuke, Hideyuki \& Haino, S \& Hams, T \& Hasegawa, Mikage \& Horikoshi, A \& Itazaki, A \& Kim, Kyung-Chan \& Kumazawa, T \& Kusumoto, A \& Lee, Min \& Makida, Y \& Matsuda, S \& Matsukawa, Y \& Matsumoto, K \& Mitchell, J.W \& Myers, Z \& Nishimura, Jun \& Nozaki, Manaho \& Yoshimura, Kenichi. \textit{Measurements of cosmic-ray proton and helium spectra from the BESS-Polar long-duration balloon flights over Antarctica.} The Astrophysical Journal. (2015). 822. 10.3847/0004-637X/822/2/65. 

\bibitem{hathaway}
  D. H. Hathaway, The Solar Cycle, Living Reviews in Solar Physics, 12 (2015).

\bibitem{forbush-decrease}
  U. Tortermpun, D. Ruffolo, and J. W. Bieber, “Galactic Cosmic-Ray Anistropy During the Forbush Decrease Starting 2013 April 13,” The Astrophysical Journal, 852, (2018).
  
\bibitem{regener}
  Regener E. \& Pfotzer G., \textit{Vertical Intensity of Cosmic Rays by Threefold Coincidences in the Stratosphere.}, Nature 136, 718-719, (1935). 

\bibitem{nasa-dose}
  C. W. Lloyd, S. K. Townsend, and K. K. Reeves. \textit{Space Radiation}. NASA, n.d.

\bibitem{ncrp}
National Council on Radiation Protection and Measurements (NCRP), \textit{Ionizing Radiation Exposure of the Population of the United States: (Report No. 160)}, (2009).
  
\bibitem{faa}
  W. Friedberg, K. Copeland, What Aircrews Should Know About Their Occupational Exposure to Ionizing Radiation, available at \url{https://www.faa.gov/data_research/research/med_humanfacs/oamtechreports/2000s/media/0316.pdf} (accessed on February 1st, 2019).
    
\bibitem{flightatt}
  B. Grajewski, E. A. Whelan, C. C. Lawson, M. J. Hein, et al., Miscarriage Among Flight Attendants, Epidemiology, 26 (2015) 192–203.

\bibitem{nasa-reid}
  NASA, NASA Space Flight Human-System Standard Volume. 1, Revision A: Crew health, (2007).
  
\bibitem{crater}
  N. A. Schwadron, J. B. Blake, A. W. Case, C. J. Joyce, et al., Does the worsening galactic cosmic radiation environment observed by CRaTER preclude future manned deep space exploration? Space Weather, 12, (2014) 622 - 632.

\bibitem{bexus}
  J. Urbar, J. Scheirich, J. Jakubek, Medipix/Timepix cosmic ray tracking on BEXUS stratospheric balloonflights. Nucl. Instrum. Methods A, 633 (2011) 206 - 209.

\bibitem{timepixiss} N. Stoffle, L. Pinsky, M. Kroupa, S. Hoang, et al., Timepix-based radiation environment monitor measurements aboard the International Space Station. Nucl. Instrum. Methods Phys. Res., A, 782 (2015) 143 - 148.

%\bibitem{TimePIX} source for image: Measurement of the energy resolution and calibration of hybrid pixel detectors with GaAs:Cr sensor and Timepix readout chip - Scientific Figure on ResearchGate. Available from: \url{https://www.researchgate.net/figure/a-Structure-of-a-Timepix-detector-b-Timepix-pixel-detector-with-USB-interface-FITPix_fig3_270906129} [accessed 23 Apr, 2019]
%%%%%Format this properly before turning this in! Note to self
\bibitem{advacam}
  ADVACAM, available at \url{http://www.advacam.com} (accessed on February 2nd, 2017).
  
\bibitem{hasp}
  High Altitude Student Platform, available at \url{http://laspace.lsu.edu/hasp/} (accessed on April 21, 2019).
  
\bibitem{rpi}
  Raspberry Pi Foundation, Raspberry Pi, available at \url{https://www.raspberrypi.org/} (accessed on August 23, 2019).

\bibitem{mpjakubek}
  J. Jakubek, Precise energy calibration of pixel detector working in time-over-threshold mode, Nucl. Instrum. Methods Phys. Res., A, 633 (2011) 262 - 266.

%\bibitem{minipix}MiniPIX - Miniaturized Portable USB Photon Counting Camera, available at \url{http://advacam.com/camera/minipix} (accessed February 2nd, 2019).
\bibitem{cosmic_flux}
The Worsening Cosmic Ray Situation, available at \url{https://spaceweatherarchive.com/2018/03/05/the-worsening-cosmic-ray-situation/} (accessed on March 6th, 2019).  

\bibitem{modeling} C. J. Mertens, Overview of the Radiation Dosimetry Experiment (RaD‐X) flight mission, Space Weather, 14 (2018) 921 – 924.  
  
\bibitem{human_impact} N. A. Schwadron, F. Rahmanifard, J. WilsonB., A. P. Jordan et al., Update on the Worsening Particle Radiation Environment Observed by CRaTER and Implications for Future Human Deep‐Space Exploration, Space Weather, 16 (2018) 289 – 303.  
  
\bibitem{aircrewexposure} B. J. Lewis, M. J. McCall, A. R. Green, L. G. I. Bennett, M. Pierre, U. J. Schrewe, K. O'Brien, E. Felsberger, Aircrew Exposure from Cosmic Radiation on Commercial Airline Routes, Radiation Protection Dosimetry, 93 (2001) 293 – 314.
  
\bibitem{timepixdosimetry} C. Granja, S. Pospisil, Quantum dosimetry and online visualization of X-ray and charged particle radiation in commercial aircraft at operational flight altitudes with the pixel detector Timepix, Advances in Space Research, 54 (2014) 241 - 251.

\bibitem{medipix}
  Medipix collaboration, available at \url{https://medipix.web.cern.ch/} (accessed on April 21, 2019).

\bibitem{jakubek-pixel-detectors}
  J. Jakubek, Semiconductor Pixel detectors and their applications in life sciences, JINST 4, (2009).
  
\bibitem{stuartthesis}
  S. George, Dosimetric Applications of Hybrid Pixel Detectors, available at \url{https://ro.uow.edu.au/cgi/viewcontent.cgi?article=5605&context=theses} (accessed on April 21, 2019).

\bibitem{mpdatasheet}
  ADVACAM, MINIPIX Version 1.0 Datasheet, available at \url{http://advacam.com/system/wp-content/uploads/2017/05/MiniPIX-Datasheet-v2-45fps-2017-11-08.pdf} (accessed on April 21, 2019).  

\bibitem{medipix-neutron-scintillator-1}
  P. Masek, J. Jakubek, J. Uher, R. Prestond, Directional detection of fast neutrons by the Timepix pixel detector coupled to plastic scintillator with silicon photomultiplier array, Journal of Instrumentation, 8 (2013).

\bibitem{medipix-neutron-scintillator-2}
  J. Uher, Ch. Fröjdh, T. Holý, J. Jakůbek, S. Petersson, S. Pospíšif, G. Thungström, D. Vavřík, and Z. Vykydal, Silicon Detectors for Neutron Imaging, AIP Conference Proceedings, 958 (2007) 101 - 104.

\bibitem{rad-x}
  A. D. P. Hands, K. A. Ryden, C. J. Mertens, The disappearance of the pfotzer-regener maximum in dose equivalent measurements in the stratosphere, Space Weather, 14 (2016) 776 - 785.

\end{thebibliography}
%
%
%
%%References and how to set them up.  From https://www.elsevier.com/journals/nuclear-instruments-and-methods-in-physics-research-section-a-accelerators-spectrometers-detectors-and-associated-equipment/01689002/guide-for-authors
%
%Journal Publication:
%[1] T. Ogi, D. Hidayat, F. Iskandar, A. Purwant, K. Okuyama, Direct synthesis of highly crystalline transparent conducing oxide nanoparticles by low pressure spray pyrolysis, Advanced Powder Technology, 20 (2009) 203 - 209.
%
%Book:
%[2] R.B. Bird, W.E. Stewart, E.N. Lightfoot, Transport Phenomena, 2nd ed., John Wiley & Sons, New York, 2002.
%
%Edited Book:
%[3] H. Masuda, K. Higashitani, H. Yoshida Eds., Powder Technology Handbook, 3rd ed., CRC Press, Boca Raton, 2006.
%
%Book Section:
%[4] Y. Mori, K. Kimura, M. Tanigaki, Influence of zone broadening on particle size analysis by sedimentation field-flow fractionation, in: N.G.
%Stanley-Wood, R.W. Lines (Eds.) Particle Size Analysis, Redwood Press, Melksham, 1992, pp. 290-299.
%
%Proceedings:
%[5] W. Ducker, N. Nicholas, G. Franks, Surface of ZnO during hydrothermal growth, in: Proc. 241st ACS National Meeting & Exposition American Chemical Society, Anaheim, CA, United States, 2011, COLL-478.
%
%Patent:
%[6] H.D. Jang, K. Cho, B.-G. Kim (Korea Institute of Geoscience and Mineral Resource, S. Korea), US20100048741A1, 2010.
%
%Web Reference:
%[7] D. Kriesel, A Brief Introduction to Neural Networks, available at http://www.dkriesel.com (accessed on January 1st, 2010).
%
%Data Citation:
%[8] M. Oguro, S. Imahiro, S. Saito, T. Nakashizuka, Mortality data for Japanese oak wilt disease and surrounding forest compositions, Mendeley Data, v1, 2015. http://dx.doi.org/10.17632/xwj98nb39r.1.
%
%Citation in text
%Please ensure that every reference cited in the text is also present in the reference list (and vice versa). Any references cited in the abstract must be given in full. Unpublished results and personal communications are not recommended in the reference list, but may be mentioned in the text. If these references are included in the reference list they should follow the standard reference style of the journal and should include a substitution of the publication date with either 'Unpublished results' or 'Personal communication'. Citation of a reference as 'in press' implies that the item has been accepted for publication.
%
%Web references
%As a minimum, the full URL should be given and the date when the reference was last accessed. Any further information, if known (DOI, author names, dates, reference to a source publication, etc.), should also be given. Web references can be listed separately (e.g., after the reference list) under a different heading if desired, or can be included in the reference list.
%
%Data references
%This journal encourages you to cite underlying or relevant datasets in your manuscript by citing them in your text and including a data reference in your Reference List. Data references should include the following elements: author name(s), dataset title, data repository, version (where available), year, and global persistent identifier. Add [dataset] immediately before the reference so we can properly identify it as a data reference. The [dataset] identifier will not appear in your published article.
%
%References in a special issue
%Please ensure that the words 'this issue' are added to any references in the list (and any citations in the text) to other articles in the same Special Issue.
%
%Reference management software
%Most Elsevier journals have their reference template available in many of the most popular reference management software products. These include all products that support Citation Style Language styles, such as Mendeley. Using citation plug-ins from these products, authors only need to select the appropriate journal template when preparing their article, after which citations and bibliographies will be automatically formatted in the journal's style. If no template is yet available for this journal, please follow the format of the sample references and citations as shown in this Guide. If you use reference management software, please ensure that you remove all field codes before submitting the electronic manuscript. 
%
%Extra References:
%
%\bibitem{SORA}S.A. Garcia Morelos, F.Brooks, S.Oliver, A.Walker, K.D. Portillo, R.B. Masek, D.Mroczek, D.Pena, J.Juarez, A.Cruz, D. Henandez, S.George, D. Pattison, A.L.Renshaw. \textit{Scientific Report for the UH Team.} SORA 2017 Mission Webpage. \textit{http://laspace.lsu.edu/hasp/groups/Payload.php?py=2017&pn=10}.
%
%
