% I suggest we discuss briefly Timepix detectors use on the ISS and other places for space radiation dosimetry and particle identification. We can also talk a bit about our motivations of studying cosmic rays on commercial airline flights.

\section{Background}
\label{Background}

TimePIX detectors have many applications in the realm of Particle Physics. Numerous balloon flights have included the use of TimePIX devices for the purposes of particle imaging and radiation dosimetry in unusual and unfamiliar environments, such as the stratosphere \cite{bexus}. The International Space Station (ISS) uses TimePIX devices to evalute the astronauts' exposure to ionizing radiation fields \cite{timepixiss}.

The TimePIX device used in this flight was a MiniPIX, which is a silicon-based hybrid pixel detector built by ADVACAM \cite{advacam}. The sensor surface offers a resolution of 256x256 pixels The device can be operated under one of three modes: time-of-arrival (TOA) or time-over-threshold (TOT), or single particle counting. When a particle ionizes with the silicon chip, the deposited charge is depleted by a bias voltage in a process known as Carrier generation and recombination. \textbf{<<Verify the previous statement>>}

\textbf{<<Include silicon-chip figure?>>}
