% I suggest we discuss briefly Timepix detectors use on the ISS and other places for space radiation dosimetry and particle identification. We can also talk a bit about our motivations of studying cosmic rays on commercial airline flights.
%
% Find model to fit our data
%
% Mention motivation for study, such as GCR's, seeing the Regener-Pfotzer Maximum, space flight and radiation importance, commercial air flights.  LOW COST RAPSBERRY PI SYSTEM.
%
\section{Background}
\label{Background}
%%% GCR's and Regener-Pfotzer Maxiumum, space flight ----- Work in Progress
Space radiation and galactic cosmic rays (GCRs) are all real danger for the space flight and aviation.  Radiation exposure in Low Earth Orbit, the Earth space radiation environment and even within the atmosphere is a constant threat. Exposure limitation is critical, so being able to measure real-time exposure can help mitigate risks.  Studying the radiation environment in the edge of space and within the atmosphere helps in preparing for current and future space missions.  

To monitor the radiation in the upper atmosphere and beyond, a USB TimePIX device could be effectively used as a low cost dosimeter.  TimePIX detectors have many applications in the realm of particle physics, specially with their small physical and power footprint. Overall, TimePIX devices could be setup to create a network of devices to better understand the Earth space radiation environment and observe space weather.
%
%
%%%   Lets tie this together, actual space below
%
%
Numerous balloon flights have included the use of TimePIX devices for the purposes of particle imaging and radiation dosimetry in unusual and unfamiliar environments, such as the stratosphere \cite{bexus}. 
%
The International Space Station (ISS) uses TimePIX devices to evaluate the astronauts' exposure to ionizing radiation fields \cite{timepix}.
% below is an actual space

The TimePIX device used in this flight was a MiniPIX, which is a silicon-based hybrid pixel detector built by ADVACAM \cite{advacam}. 
%
The sensor surface offers a resolution of 256x256 pixels.
%
The device can be operated under one of three modes: time-of-arrival (TOA) or time-over-threshold (TOT), or single particle counting. 
%
When a particle ionizes with the silicon chip, the deposited charge is depleted by a bias voltage in a process known as Carrier generation and recombination. \textbf{<<Verify the previous statement>>}
% below is an actual space

\begin{figure}[h]
    \centering
    \includegraphics[width=0.25\textwidth]{minipix_silicon.pdf}
    \caption{MiniPIX silicon concept design}
    \label{fig:minipix_silicon}
\end{figure}
%
\textbf{<<Include silicon-chip figure?>>}
%figure name: minipix_silicon.pdf
%

For the SORA flights, the MiniPIX was coupled with a low cost open ARM based computer.  Both flights utilized a Raspberry Pi 3 to communicate and handle all operations with the MiniPIX.  This system allowed for remote operation and on-board analyzation of all data.  The SORA flights tested the feasibility of these low-cost systems to fly further space missions.  