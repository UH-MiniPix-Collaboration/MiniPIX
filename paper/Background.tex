% I suggest we discuss briefly Timepix detectors use on the ISS and other places for space radiation dosimetry and particle identification. We can also talk a bit about our motivations of studying cosmic rays on commercial airline flights.

\section{Background}
\label{Background}

TimePIX detectors have many applications in the realm of Particle Physics. Numerous balloon flights have included the use of TimePIX devices for the purposes of particle imaging and radiation dosimetry in unusual and unfamiliar environments, such as the stratosphere \cite{bexus} \textbf{<<Cite ourselves?>}. The International Space Station (ISS) uses TimePIX devices to evalute the astronauts' exposure to ionizing radiation fields \cite{timepixiss}.

The authors of this paper have made arrangements to collaborate with Dr. Lawrence Pinsky and Dr. Stuart George of the University of Houston to develop a portable, TimePIX-based device capable of detecting thermal neutrons. The motivation behind this project is to investigate the production rates of and passenger exposure to thermal neutrons during commercial air flights by building off of the previous work done by groups such as B.J. Lewis et al. \cite{aircrewexposure} and C. Granga and S. Pospisil \cite{timepixdosimetry}.
